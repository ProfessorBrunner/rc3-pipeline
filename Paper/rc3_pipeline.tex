\documentclass[5p]{elsarticle}
\usepackage{natbib}
\usepackage{url}
\begin{document}
\begin{abstract}
abstract
\end{abstract}


\section{Introduction}
\section{Automation Program}
	The rc3 class creates an rc3 object containing its ``basic info" \footnote{ ra,dec,radius, pgc} for each galaxy in the RC3 catalog. The practical purpose of the class (other than the use of OOP that actually makes sense) is to keep tract of the number of iterations in the recursive step using the instance variable num\_iterations. 
	
	Some parts of the program needs to be adjusted for survey-specific , but the core concept (and bulk of the code) should stay the same. 
	%Think of a name to refer to "the program"

		\subsection{Technical Details}
	The program is written in Python 2.7.6
 It uses IPAC's Montage  \cite{montage} using the AstroPy Montage wrapper\footnote{\url{http://www.astropy.org/montage-wrapper/}}. The final g,r,i image is created using Astromatic STIFF \cite{stiff} . Our choice of program reflects best feature from both program : STIFF provides the flexibility of changing many variables for the final g,r,i; Montage creates scientifically-calibrated images by retaining astrometric accuracy during image reprojection. The use of two different program in the mosaicing step is due to Montage's ability to create scientifically-calibrated mosaic FITS, but STIFF provided more flexible parameters for combining all bands into color images, so we get the best of both worlds. The STIFF setting needs to be adjusted for each survey depending on specs on the telescope's CCD dependent parameters such as imaging bands. (?) The source extraction is done using SExtractor. If the mosaicing field is chosen correctly, then SExtractor's skylevel estimation is fairly accurate.
	\subsection{Getting the Data}
		\subsubsection{Search}
			Figure out how to convert positional values (ra,dec) to record-keeping parameters dependent on the survey's telescope. (tiles, frames...etc) \footnote{Since SDSS images are taken in drift-scan mode, we need the run,camcol,field values to acess each iamge frame} Usually this step can be done using the SQL search. 
Most surveys have an API that enables you to acess data using SQL querys so that the mosaicing program can interact with so that it doesn't have to  click buttons or type in textboxes in the web GUI. 
\section{Algorithms}
\section{Pipeline}
\section{Known Errors}


%20.9170833333     33.2561111111     0.0539322762628      5098   
%20.9191666667     33.2811111111     0.0230064044561      5099 
\section{Algorithm}

%\bibliographystyle{elsarticle-num}
%\bibliographystyle{te}
\bibliographystyle{abbrvnat}

\bibliography{bibtex_database}

\end{document}
